% Options for packages loaded elsewhere
\PassOptionsToPackage{unicode}{hyperref}
\PassOptionsToPackage{hyphens}{url}
\PassOptionsToPackage{dvipsnames,svgnames,x11names}{xcolor}
%
\documentclass[
]{ltjarticle}

\usepackage{amsmath,amssymb}
\usepackage{iftex}
\ifPDFTeX
  \usepackage[T1]{fontenc}
  \usepackage[utf8]{inputenc}
  \usepackage{textcomp} % provide euro and other symbols
\else % if luatex or xetex
  \usepackage{unicode-math}
  \defaultfontfeatures{Scale=MatchLowercase}
  \defaultfontfeatures[\rmfamily]{Ligatures=TeX,Scale=1}
\fi
\usepackage{lmodern}
\ifPDFTeX\else  
    % xetex/luatex font selection
\fi
% Use upquote if available, for straight quotes in verbatim environments
\IfFileExists{upquote.sty}{\usepackage{upquote}}{}
\IfFileExists{microtype.sty}{% use microtype if available
  \usepackage[]{microtype}
  \UseMicrotypeSet[protrusion]{basicmath} % disable protrusion for tt fonts
}{}
\makeatletter
\@ifundefined{KOMAClassName}{% if non-KOMA class
  \IfFileExists{parskip.sty}{%
    \usepackage{parskip}
  }{% else
    \setlength{\parindent}{0pt}
    \setlength{\parskip}{6pt plus 2pt minus 1pt}}
}{% if KOMA class
  \KOMAoptions{parskip=half}}
\makeatother
\usepackage{xcolor}
\setlength{\emergencystretch}{3em} % prevent overfull lines
\setcounter{secnumdepth}{-\maxdimen} % remove section numbering
% Make \paragraph and \subparagraph free-standing
\ifx\paragraph\undefined\else
  \let\oldparagraph\paragraph
  \renewcommand{\paragraph}[1]{\oldparagraph{#1}\mbox{}}
\fi
\ifx\subparagraph\undefined\else
  \let\oldsubparagraph\subparagraph
  \renewcommand{\subparagraph}[1]{\oldsubparagraph{#1}\mbox{}}
\fi

\usepackage{color}
\usepackage{fancyvrb}
\newcommand{\VerbBar}{|}
\newcommand{\VERB}{\Verb[commandchars=\\\{\}]}
\DefineVerbatimEnvironment{Highlighting}{Verbatim}{commandchars=\\\{\}}
% Add ',fontsize=\small' for more characters per line
\usepackage{framed}
\definecolor{shadecolor}{RGB}{241,243,245}
\newenvironment{Shaded}{\begin{snugshade}}{\end{snugshade}}
\newcommand{\AlertTok}[1]{\textcolor[rgb]{0.68,0.00,0.00}{#1}}
\newcommand{\AnnotationTok}[1]{\textcolor[rgb]{0.37,0.37,0.37}{#1}}
\newcommand{\AttributeTok}[1]{\textcolor[rgb]{0.40,0.45,0.13}{#1}}
\newcommand{\BaseNTok}[1]{\textcolor[rgb]{0.68,0.00,0.00}{#1}}
\newcommand{\BuiltInTok}[1]{\textcolor[rgb]{0.00,0.23,0.31}{#1}}
\newcommand{\CharTok}[1]{\textcolor[rgb]{0.13,0.47,0.30}{#1}}
\newcommand{\CommentTok}[1]{\textcolor[rgb]{0.37,0.37,0.37}{#1}}
\newcommand{\CommentVarTok}[1]{\textcolor[rgb]{0.37,0.37,0.37}{\textit{#1}}}
\newcommand{\ConstantTok}[1]{\textcolor[rgb]{0.56,0.35,0.01}{#1}}
\newcommand{\ControlFlowTok}[1]{\textcolor[rgb]{0.00,0.23,0.31}{#1}}
\newcommand{\DataTypeTok}[1]{\textcolor[rgb]{0.68,0.00,0.00}{#1}}
\newcommand{\DecValTok}[1]{\textcolor[rgb]{0.68,0.00,0.00}{#1}}
\newcommand{\DocumentationTok}[1]{\textcolor[rgb]{0.37,0.37,0.37}{\textit{#1}}}
\newcommand{\ErrorTok}[1]{\textcolor[rgb]{0.68,0.00,0.00}{#1}}
\newcommand{\ExtensionTok}[1]{\textcolor[rgb]{0.00,0.23,0.31}{#1}}
\newcommand{\FloatTok}[1]{\textcolor[rgb]{0.68,0.00,0.00}{#1}}
\newcommand{\FunctionTok}[1]{\textcolor[rgb]{0.28,0.35,0.67}{#1}}
\newcommand{\ImportTok}[1]{\textcolor[rgb]{0.00,0.46,0.62}{#1}}
\newcommand{\InformationTok}[1]{\textcolor[rgb]{0.37,0.37,0.37}{#1}}
\newcommand{\KeywordTok}[1]{\textcolor[rgb]{0.00,0.23,0.31}{#1}}
\newcommand{\NormalTok}[1]{\textcolor[rgb]{0.00,0.23,0.31}{#1}}
\newcommand{\OperatorTok}[1]{\textcolor[rgb]{0.37,0.37,0.37}{#1}}
\newcommand{\OtherTok}[1]{\textcolor[rgb]{0.00,0.23,0.31}{#1}}
\newcommand{\PreprocessorTok}[1]{\textcolor[rgb]{0.68,0.00,0.00}{#1}}
\newcommand{\RegionMarkerTok}[1]{\textcolor[rgb]{0.00,0.23,0.31}{#1}}
\newcommand{\SpecialCharTok}[1]{\textcolor[rgb]{0.37,0.37,0.37}{#1}}
\newcommand{\SpecialStringTok}[1]{\textcolor[rgb]{0.13,0.47,0.30}{#1}}
\newcommand{\StringTok}[1]{\textcolor[rgb]{0.13,0.47,0.30}{#1}}
\newcommand{\VariableTok}[1]{\textcolor[rgb]{0.07,0.07,0.07}{#1}}
\newcommand{\VerbatimStringTok}[1]{\textcolor[rgb]{0.13,0.47,0.30}{#1}}
\newcommand{\WarningTok}[1]{\textcolor[rgb]{0.37,0.37,0.37}{\textit{#1}}}

\providecommand{\tightlist}{%
  \setlength{\itemsep}{0pt}\setlength{\parskip}{0pt}}\usepackage{longtable,booktabs,array}
\usepackage{calc} % for calculating minipage widths
% Correct order of tables after \paragraph or \subparagraph
\usepackage{etoolbox}
\makeatletter
\patchcmd\longtable{\par}{\if@noskipsec\mbox{}\fi\par}{}{}
\makeatother
% Allow footnotes in longtable head/foot
\IfFileExists{footnotehyper.sty}{\usepackage{footnotehyper}}{\usepackage{footnote}}
\makesavenoteenv{longtable}
\usepackage{graphicx}
\makeatletter
\def\maxwidth{\ifdim\Gin@nat@width>\linewidth\linewidth\else\Gin@nat@width\fi}
\def\maxheight{\ifdim\Gin@nat@height>\textheight\textheight\else\Gin@nat@height\fi}
\makeatother
% Scale images if necessary, so that they will not overflow the page
% margins by default, and it is still possible to overwrite the defaults
% using explicit options in \includegraphics[width, height, ...]{}
\setkeys{Gin}{width=\maxwidth,height=\maxheight,keepaspectratio}
% Set default figure placement to htbp
\makeatletter
\def\fps@figure{htbp}
\makeatother

\usepackage{luatexja-fontspec}
\makeatletter
\makeatother
\makeatletter
\makeatother
\makeatletter
\@ifpackageloaded{caption}{}{\usepackage{caption}}
\AtBeginDocument{%
\ifdefined\contentsname
  \renewcommand*\contentsname{目次}
\else
  \newcommand\contentsname{目次}
\fi
\ifdefined\listfigurename
  \renewcommand*\listfigurename{図一覧}
\else
  \newcommand\listfigurename{図一覧}
\fi
\ifdefined\listtablename
  \renewcommand*\listtablename{表一覧}
\else
  \newcommand\listtablename{表一覧}
\fi
\ifdefined\figurename
  \renewcommand*\figurename{図}
\else
  \newcommand\figurename{図}
\fi
\ifdefined\tablename
  \renewcommand*\tablename{表}
\else
  \newcommand\tablename{表}
\fi
}
\@ifpackageloaded{float}{}{\usepackage{float}}
\floatstyle{ruled}
\@ifundefined{c@chapter}{\newfloat{codelisting}{h}{lop}}{\newfloat{codelisting}{h}{lop}[chapter]}
\floatname{codelisting}{コード}
\newcommand*\listoflistings{\listof{codelisting}{コード一覧}}
\makeatother
\makeatletter
\@ifpackageloaded{caption}{}{\usepackage{caption}}
\@ifpackageloaded{subcaption}{}{\usepackage{subcaption}}
\makeatother
\makeatletter
\@ifpackageloaded{tcolorbox}{}{\usepackage[skins,breakable]{tcolorbox}}
\makeatother
\makeatletter
\@ifundefined{shadecolor}{\definecolor{shadecolor}{rgb}{.97, .97, .97}}
\makeatother
\makeatletter
\makeatother
\makeatletter
\makeatother
\ifLuaTeX
\usepackage[bidi=basic]{babel}
\else
\usepackage[bidi=default]{babel}
\fi
\babelprovide[main,import]{japanese}
% get rid of language-specific shorthands (see #6817):
\let\LanguageShortHands\languageshorthands
\def\languageshorthands#1{}
\ifLuaTeX
  \usepackage{selnolig}  % disable illegal ligatures
\fi
\IfFileExists{bookmark.sty}{\usepackage{bookmark}}{\usepackage{hyperref}}
\IfFileExists{xurl.sty}{\usepackage{xurl}}{} % add URL line breaks if available
\urlstyle{same} % disable monospaced font for URLs
\hypersetup{
  pdftitle={Japanese Education Statistics Dataset},
  pdfauthor={Beatriz Gietner},
  pdflang={ja-JP},
  colorlinks=true,
  linkcolor={blue},
  filecolor={Maroon},
  citecolor={Blue},
  urlcolor={Blue},
  pdfcreator={LaTeX via pandoc}}

\title{Japanese Education Statistics Dataset}
\author{Beatriz Gietner}
\date{}

\begin{document}
\maketitle
\ifdefined\Shaded\renewenvironment{Shaded}{\begin{tcolorbox}[enhanced, frame hidden, breakable, sharp corners, boxrule=0pt, interior hidden, borderline west={3pt}{0pt}{shadecolor}]}{\end{tcolorbox}}\fi

\hypertarget{introduction}{%
\section{Introduction}\label{introduction}}

This document provides information on a set of comprehensive education
statistics for Japan, covering various aspects of the education system
from early childhood education to university level. The data is at the
prefecture level, with an additional row for national averages, forming
an unbalanced panel dataset.

\begin{Shaded}
\begin{Highlighting}[]
\NormalTok{knitr}\SpecialCharTok{::}\NormalTok{opts\_chunk}\SpecialCharTok{$}\FunctionTok{set}\NormalTok{(}\AttributeTok{echo =} \ConstantTok{TRUE}\NormalTok{)}
\CommentTok{\# Clean environment}
\FunctionTok{rm}\NormalTok{(}\AttributeTok{list =} \FunctionTok{ls}\NormalTok{())}
\CommentTok{\# Load necessary libraries}
\FunctionTok{library}\NormalTok{(tidyverse)}
\FunctionTok{library}\NormalTok{(knitr)}
\FunctionTok{library}\NormalTok{(ggplot2)}
\FunctionTok{library}\NormalTok{(readxl)}
\end{Highlighting}
\end{Shaded}

\begin{Shaded}
\begin{Highlighting}[]
\NormalTok{knitr}\SpecialCharTok{::}\NormalTok{opts\_chunk}\SpecialCharTok{$}\FunctionTok{set}\NormalTok{(}\AttributeTok{echo =} \ConstantTok{TRUE}\NormalTok{)}
\CommentTok{\# Load dataset}
\CommentTok{\# japan\_ed\_data \textless{}{-} read\_csv("path/to/your/data.csv") }
\CommentTok{\# or}
\CommentTok{\# japan\_ed\_data \textless{}{-} read\_excel("path/to/your/data.xlsx")}
\end{Highlighting}
\end{Shaded}

\hypertarget{data-overview}{%
\section{Data Overview}\label{data-overview}}

\hypertarget{variables-description}{%
\subsection{Variables Description}\label{variables-description}}

The dataset includes the following key variables:

\begin{itemize}
\tightlist
\item
  Prefecture\_code: Numeric identifier for each prefecture
\item
  Prefecture\_jp: Name of the prefecture in Japanese
\item
  Prefecture\_en: Name of the prefecture in English
\item
  Year: Year of observation
\item
  Value: Value of the specific educational statistic for that prefecture
  and year
\end{itemize}

\hypertarget{geographical-scope}{%
\subsection{Geographical Scope}\label{geographical-scope}}

The data covers all 47 prefectures of Japan, plus a national average, a
mean 平均値 and its standard deviation 標準偏差.

\hypertarget{sections-overview}{%
\section{Sections Overview}\label{sections-overview}}

The dataset is an unbalanced panel, with data points starting in 2005,
jumping to 2010, and then continuing from 2012 to 2022 for most
variables. Each sheet in the original Excel file represents a different
educational statistic, including:

\begin{enumerate}
\def\labelenumi{\arabic{enumi}.}
\tightlist
\item
  Educational institutions 教育施設
\item
  Teachers 教員
\item
  Pupils 生徒
\item
  Junior colleges, colleges, and universities 短期大学・大学
\item
  Specialty schools and miscellaneous schools 専修学校・各種学
\item
  Educational diffusion rates 教育普及度
\item
  Education expenditure per pupil 1人当たりの学校等教育費
\end{enumerate}

Each variable is represented in a sheet according to the order above.
They contains the name in both in English and Japanese.

The data for each statistic is presented in a long format, with all
variables as rows. The dataset includes 47 prefectures plus a row for
the national average 全国.

Note: The actual number of variables and observations may differ from
the placeholder values in this documentation. Users should check the
loaded data for the most accurate information.

For each section, it is provided: a) a list of variables,\\
b) the years covered,\\
c) the calculation methods, and~ d) an interpretation of what the
metrics mean and their significance.

\hypertarget{section-1-education-ux6559ux80b2-educational-institutions-ux6559ux80b2ux65bdux8a2d}{%
\subsection{Section 1: Education 教育, Educational institutions
教育施設:}\label{section-1-education-ux6559ux80b2-educational-institutions-ux6559ux80b2ux65bdux8a2d}}

The variables covered in this section are:

1.1 Elementary schools (per 100,000 population 6-11 years)\\
1.2 Junior high schools (per 100,000 population 12-14 years)\\
1.3 Senior high schools (per 100,000 population 15-17 years)\\
1.4 Kindergartens (per 100,000 population 3-5 years)\\
1.5 Day nurseries (per 100,000 population 0-5 years)\\
1.6 Authorized child centers (per 100,000 population 0-5 years)

2.1 Elementary schools (per inhabitable area 100 k㎡)\\
2.2 Junior high schools (per inhabitable area 100 k㎡) pre-2019 / Number
of lower secondary schools (per inhabitable area 100 k㎡) from 2019
onwards\\
2.3 Senior high schools (per inhabitable area 100 k㎡) pre-2019 / Number
of upper secondary schools (per inhabitable area 100 k㎡) from 2019
onwards

3.1 Ratio/percentage of public senior high/upper secondary schools\\
3.2 Ratio/percentage of public kindergartens\\
3.3 Ratio/percentage of public day nurseries

\hypertarget{subsections-1.1---1.6}{%
\subsubsection{Subsections 1.1 - 1.6:}\label{subsections-1.1---1.6}}

The data is separated in columns per 100,000 population by different age
ranges:\\
Elementary schools 小学校数: 6-11\\
Junior high schools 中学校数: 12-14\\
Senior high schools 高等学校数: 15-17\\
Kindergartens 幼稚園数: 0-3\\

\begin{itemize}
\tightlist
\item
  The years covered are: 2005, 2010, 2012-2020
\end{itemize}

Day nurseries 保育所数: 3-5\\

\begin{itemize}
\item
  The years covered are: 2005, 2010-2018, 2020
\item
  The calculations are made by employment the formula: (Number of
  ``variable'' / Population aged ``variable''``) * 100,000
\item
  For example, if we take data for Kanagawa 神奈川県 in 2012 we have
  184.9, it was calculated by (Number of elementary schools in Kanagawa
  / Population of 6-11 year olds in Kanagawa) * 100,000 = 184.9. This
  method allows for comparison between prefectures with different
  population sizes.
\end{itemize}

One variable that was introduced later in the dataset was: Authorized
child centers (per 100,000 population 0-5 years)
認定こども園数(0~5歳人口10万人当たり)\\

\begin{itemize}
\item
  The years covered are 2015, 2018-2022
\item
  The calculations are made by employing the formula: (Number of
  authorized child centers / Population aged 0-5 years) * 100,000
\item
  For each prefecture and year, they would have: counted the number of
  authorized child centers, counted the total population aged 0-5 years,
  divided the number of centers by the population, then multiplied the
  result by 100,000 to get the ratio per 100,000 children.
\item
  This metric provides insights into the availability of authorized
  child centers across different regions of Japan relative to the young
  child population. It can reflect various factors such as local
  childcare policies, demographic trends, and societal needs in
  different areas. The data over time allows for analysis of trends in
  the expansion of authorized child centers. These figures can be
  important indicators of early childhood education and care provision,
  which may have implications for child development, work-life balance
  for parents, and broader social policies. Comparing the ratios between
  prefectures can reveal regional differences in childcare
  infrastructure and potentially inform resource allocation and policy
  decisions.
\end{itemize}

\hypertarget{subsections-2.1---2.3}{%
\subsubsection{Subsections 2.1 - 2.3}\label{subsections-2.1---2.3}}

They contain information on Junior high schools 中学校数-Lower secondary
schools and Senior high schools 高等学校数-Upper secondary schools per
inhabitable area 100 k㎡ 可住地面積100k㎡当たり.

\begin{itemize}
\item
  The years covered are 2005, 2010, 2012-2022
\item
  The calculations are made by employment the formula: (Number of
  ``variable schools'' / Inhabitable area in k㎡) * 100
\item
  For each prefecture and year, they would have: counted the total
  number of elementary schools, measured the total inhabitable area in
  square kilometers, divided the number of schools by the area, and then
  multiplied the result by 100 to get the number per 100 k㎡.
\item
  This measure shows how densely or sparsely schools are distributed
  across the habitable land in each prefecture, which can be
  particularly relevant in a country like Japan with varying population
  densities and significant uninhabitable areas (like mountains).
\end{itemize}

\hypertarget{subsections-3.1---3.3}{%
\subsubsection{Subsections 3.1 - 3.3:}\label{subsections-3.1---3.3}}

They contain information on the Ratio/percentage of public senior
high/upper secondary schools 公立高等学校割合 (\%), Ratio/percentage of
public kindergartens 公立幼稚園割合 (\%), and Ratio/percentage of public
day nurseries/nurse centers 公営保育所割合 (\%) for the years 2005,
2010, 2012-2022 (except for the last one, Ratio/percentage of public day
nurseries/nurse centers, for which we have data for the years 2005,
2010-2018, 2020).

\begin{itemize}
\item
  The calculations are made by employment the formula: (Number of public
  ``variable schools'' / Total number of ``variable schools'') * 100
\item
  For each prefecture and year, they would have: counted the number of
  public schools, counted the total number schools, divided the number
  of public schools by the total number, then multiplied the result by
  100 to get the percentage. For example, the national average 全国 for
  2012 is 73.4\%, which means that in 2012, 73.4\% of all senior high
  schools in Japan were public schools. It was named ``ratio'' until
  2020, when the variable was changed to ``percentage'', but they were
  calculated the same way.
\item
  These metrics collectively provide a comprehensive view of the
  educational infrastructure across Japan. The number of institutions
  per population gives insight into the accessibility of education for
  different age groups, while the number per inhabitable area reflects
  the physical distribution of schools. The ratios/percentages of public
  institutions show the balance between public and private education
  provision. Together, these metrics can reveal regional differences in
  educational resources, urbanization patterns, and government
  priorities in education. They allow for comparisons between
  prefectures with different population sizes and geographical
  characteristics. Analyzing these metrics over time (2005, 2010,
  2012/2011) can also indicate trends in educational policy and
  infrastructure development across Japan, such as changes in school
  distribution or shifts in the balance between public and private
  education. This information is important for understanding the
  educational landscape and informing policy decisions related to
  educational access and resource allocation.
\end{itemize}

\hypertarget{section-2-education-ux6559ux80b2-teachers-ux6559ux54e1}{%
\subsection{Section 2: Education 教育, Teachers
教員:}\label{section-2-education-ux6559ux80b2-teachers-ux6559ux54e1}}

The variables covered in this section are:

Ratio of female teachers (elementary school)\\
Ratio of female teachers (junior high/lower secondary school)

\hypertarget{subsections-1.1---1.2}{%
\subsubsection{Subsections 1.1 - 1.2:}\label{subsections-1.1---1.2}}

They contain information on the Ratio of female teachers 女子教員割合
(\%) for elementary 小学校 and junior high/lower secondary school school
中学校.

\begin{itemize}
\item
  The years covered are 2005, 2010, 2012-2022
\item
  The calculations are made by employment the formula: (Number of female
  ``variable school'' teachers / Total number of ``variable school''
  school teachers) * 100
\item
  For each prefecture and year, they would have: counted the number of
  female ``variable school'' teachers, counted the total number of
  ``variable school'' teachers, divided the number of female teachers by
  the total number, then multiplied the result by 100 to get the
  percentage.
\item
  These metrics provide insights into the gender balance among school
  teachers across different regions of Japan. They can reflect various
  factors such as societal norms, employment policies, and career
  preferences in different areas. Comparing the ratios between
  elementary and junior high/lower secondary school schools can reveal
  differences in gender distribution at different levels of education.
  The data over time allows for analysis of trends in gender
  representation in the teaching profession. These figures can be
  important indicators of gender equality in the education sector and
  may have implications for students' perceptions of gender roles in
  society.
\end{itemize}

\hypertarget{section-3-education-ux6559ux80b2-pupils-ux751fux5f92}{%
\subsection{Section 3: Education 教育, Pupils
生徒:}\label{section-3-education-ux6559ux80b2-pupils-ux751fux5f92}}

The variables covered in this section are:

1.1 Elementary school pupils (per teacher)\\
1.2 Junior high school students (per teacher)\\
1.3 Senior high school students (per teacher)\\
1.4 Kindergarten pupils (per teacher)\\
1.5 Day nursery pupils (per nurse)

2.1 Ratio of public senior high school students\\
2.2 Ratio of public kindergarteners\\
2.3 Ratio of public day nursery children

3.1 Elementary school pupils (per class)\\
3.2 Junior high school students (per class)

\hypertarget{subsections-1.1---1.5}{%
\subsubsection{Subsections 1.1 - 1.5:}\label{subsections-1.1---1.5}}

They contain information on the number of Elementary school pupils
小学校児童数, Junior high school students 中学校生徒数, Senior high
school students 高等学校生徒数, and Kindergarten pupils 幼稚園在園者数,
all per teacher 教員1人当た.

\begin{itemize}
\tightlist
\item
  The years covered are 2005, 2010, 2012-2022
\end{itemize}

And Day nursery pupils 保育所在所児数 per nurse 保育士1人当たり.

\begin{itemize}
\item
  The years covered are 2005, 2010-2018, 2020
\item
  The calculations are made by employing the formula: (Total number of
  ``variable school'' students/pupils) / (Total number of ``variable
  school'' teachers/nurses)
\item
  For each prefecture and year, they would have: counted the total
  number of ``variable school'' students/pupils, counted the total
  number of ``variable school'' teachers/nurses, then divided the number
  of students/pupils by the number of teachers/nurses.
\end{itemize}

\hypertarget{subsections-2.1---2.3-1}{%
\subsubsection{Subsections 2.1 - 2.3:}\label{subsections-2.1---2.3-1}}

They contain information on the Ratio of public senior high school
公立高等学校生徒比率 (\%), and Ratio of public kindergarteners
公立幼稚園在園者比率 (\%).

\begin{itemize}
\tightlist
\item
  The years covered are 2005, 2010, 2012-2022
\end{itemize}

They contain information on the Ratio of public day nursery children
公営保育所在所児比率 (\%).

\begin{itemize}
\item
  The years covered are 2005, 2010-2018, 2020-2021
\item
  The calculations are made by employing the formula: (Number of
  students/pupils/children in public ``variable schools'' / Total number
  of ``variable school'' students/pupils/children) * 100
\item
  For each prefecture and year, they would have: counted the number of
  students/pupils/children in public ``variable school'', counted the
  total number of students/pupils/children (public and private), divided
  the number of public ``variable school'' students/pupils/children by
  the total, then multiplied the result by 100 to get the percentage.
\end{itemize}

\hypertarget{subsections-3.1---3.2}{%
\subsubsection{Subsections 3.1 - 3.2:}\label{subsections-3.1---3.2}}

They contain information on the number of Elementary school pupils per
class 小学校児童数 (1学級当たり, 人), and Junior high school students
中学校生徒数 (1学級当たり, 人).

\begin{itemize}
\item
  The years covered are 2005, 2010, 2012-2022
\item
  The calculations are made by employing the formula: (Total number of
  ``variable school'' students) / (Total number of classes in ``variable
  school'')
\item
  For each prefecture and year, they would have: counted the total
  number of ``variable school'' students, counted the total number of
  classes in ``variable school'', then divided the number of students by
  the number of classes.
\item
  These metrics provide a multifaceted view of the education system in
  Japan, from early childhood through high school. The
  student-to-teacher and student-to-class ratios offer insights into the
  resources available to students and the potential quality of
  individual attention they might receive. The ratios of public
  institution attendance reflect the balance between public and private
  education in different regions, which can indicate both government
  priorities and societal preferences. Together, these metrics can
  reveal regional differences in educational resources, class sizes, and
  the structure of the education system. Comparing these metrics over
  time can also indicate trends in educational policy and practice
  across Japan, such as efforts to reduce class sizes or shifts in the
  balance between public and private education.
\end{itemize}

\hypertarget{section-4-education-ux6559ux80b2-junior-colleges-colleges-and-universities-ux77edux671fux5927ux5b66ux5927ux5b66}{%
\subsection{Section 4: Education 教育, Junior colleges, colleges and
universities
短期大学・大学:}\label{section-4-education-ux6559ux80b2-junior-colleges-colleges-and-universities-ux77edux671fux5927ux5b66ux5927ux5b66}}

The variables covered in this section are:

1.1 Junior colleges (per 100,000 persons)\\
1.2 Colleges and universities (per 100,000 persons)\\

2.1 Entrance capacity index of junior colleges\\
2.2 Entrance capacity index of colleges and universities

3.1 Ratio of students of national colleges + universities to all
students\\
3.2 Ratio of students of public colleges and universities to all
students\\
3.3 Ratio of students of private colleges and universities to all
students

\hypertarget{subsections-1.1---1.2-1}{%
\subsubsection{Subsections 1.1 - 1.2:}\label{subsections-1.1---1.2-1}}

They contain information on the Number of Junior Colleges (per 100,000
persons) 短期大学数 (人口10万人当たり), Number of Colleges and
Universities (per 100,000 persons) 大学数 (人口10万人当たり).

\begin{itemize}
\item
  The years covered are 2005, 2010, 2012-2022
\item
  The calculations are made by employing the formula: (Number of
  institutions) / (Total population) * 100,000
\item
  For each prefecture and year, they would have: counted the number of
  institutions, obtained the total population, divided the number of
  institutions by the population, then multiplied the result by 100,000.
\end{itemize}

\hypertarget{subsections-2.1---2.2}{%
\subsubsection{Subsections 2.1 - 2.2:}\label{subsections-2.1---2.2}}

They contain information on Entrance Capacity Index of Junior Colleges
短期大学収容力指数 and Entrance Capacity Index of Colleges and
Universities 大学収容力指数.

\begin{itemize}
\item
  The years covered are 2005, 2010, 2012-2022
\item
  The calculations are made by employing the formula: (Number of new
  entrants × 100) / (Number of high school graduates who advanced to
  junior colleges or universities in the previous year)
\item
  For each prefecture and year, they would have: counted the number of
  new entrants, counted the number of high school graduates who advanced
  to junior colleges or universities in the previous year, divided the
  number of new entrants by the number of advancing graduates, then
  multiplied the result by 100.
\end{itemize}

\hypertarget{subsections-3.1---3.3-1}{%
\subsubsection{Subsections 3.1 - 3.3:}\label{subsections-3.1---3.3-1}}

They contain information on the Ratio (\%) of students in National
Colleges and Universities 国立大学学生数割合, Ratio of students in
Public Colleges and Universities 公立大学学生数割合, and Ratio of
students in Private Colleges and Universities 私立大学学生数割合.

\begin{itemize}
\item
  The years covered are 2005, 2010, 2012-2022
\item
  The calculations are made by employing the formula: (Number of
  students in national/public/private institutions) / (Total number of
  university students) * 100
\item
  For each prefecture and year, they would have: counted the number of
  students in national, public, or private institutions, counted the
  total number of university students, divided the number of students in
  each category by the total, then multiplied the result by 100 to get
  the percentage.
\item
  These metrics provide a comprehensive view of higher education in
  Japan. The number of institutions per capita indicates the
  accessibility of higher education in different regions. The entrance
  capacity index shows how well the supply of higher education spots
  meets the demand from high school graduates. The ratios of students in
  national, public, and private institutions reflect the structure of
  the higher education system in each prefecture. Together, these
  metrics can reveal regional differences in educational opportunities,
  the balance between different types of institutions, and potentially,
  the educational priorities or strategies of different prefectures.
  Comparing these metrics over time can also indicate trends in the
  development of Japan's higher education system.
\end{itemize}

\hypertarget{section-5-education-ux6559ux80b2-speciality-schools-and-miscellaneous-schools-ux5c02ux4feeux5b66ux6821ux5404ux7a2eux5b66ux6821}{%
\subsection{Section 5: Education 教育, Speciality schools and
miscellaneous schools
専修学校・各種学校:}\label{section-5-education-ux6559ux80b2-speciality-schools-and-miscellaneous-schools-ux5c02ux4feeux5b66ux6821ux5404ux7a2eux5b66ux6821}}

The variables covered in this section are:

1.1 Speciality schools (per 100,000 persons)\\
1.2 Miscellaneous schools (per 100,000 persons)

2.1 Students enrolled in speciality schools (per 1,000 persons)/Number
of specialized training college students (per 1,000 persons)\\
2.2 Students enrolled in miscellaneous schools (per 1,000 persons)

\hypertarget{subsections-1.1---1.2-2}{%
\subsubsection{Subsections 1.1 - 1.2:}\label{subsections-1.1---1.2-2}}

They contain information on the Number of Speciality Schools per 100,000
persons 専修学校数 (人口10万人当たり)/Number of specialized training
college students (new English nomenclature in 2020/2021/2022), Number of
Miscellaneous Schools per 100,000 persons 各種学校数 (人口10万人当たり).

\begin{itemize}
\item
  The years covered are 2005, 2010, 2012-2022
\item
  The calculations are made by employing the formula: (Number of
  schools) / (Total population) * 100,000.
\item
  For each prefecture and year, they would have: counted the number of
  schools, obtained the total population, divided the number of schools
  by the population, then multiplied the result by 100,000.
\item
  These calculations provide insights into the prevalence and
  utilization of speciality and miscellaneous schools across Japan.
  Speciality schools 専修学校 typically offer practical vocational
  education, while miscellaneous schools 各種学校 cover a wide range of
  educational institutions that don't fit into other categories, such as
  international schools or schools for specific skills like cooking or
  languages. The rates of schools and enrolled students per capita can
  indicate the diversity of educational options available in different
  prefectures, the demand for specialized or vocational education, and
  potentially reflect regional economic needs or cultural preferences
  for different types of education.
\end{itemize}

\hypertarget{subsections-2.1---2.2-1}{%
\subsubsection{Subsections 2.1 - 2.2:}\label{subsections-2.1---2.2-1}}

They contain information on the number of Students enrolled in
Speciality Schools per 1,000 persons 専修学校生徒数 (人口千人当たり) and
Students enrolled in Miscellaneous Schools per 1,000 persons
各種学校生徒数 (人口千人当たり).

\begin{itemize}
\item
  The years covered are 2005, 2010, 2012-2022
\item
  The calculations are made by employing the formula: (Number of
  enrolled students) / (Total population) * 1,000
\item
  For each prefecture and year, they would have: counted the number of
  enrolled students, obtained the total population, divided the number
  of students by the population, then multiplied the result by 1,000.
\item
  These calculations allow for comparisons of speciality and
  miscellaneous schools across prefectures and over time. They provide
  insights into the availability and popularity of these types of
  educational institutions in different regions of Japan, taking into
  account the population differences between prefectures.
\end{itemize}

\hypertarget{section-6-education-ux6559ux80b2-educational-diffusion-rates-ux6559ux80b2ux666eux53caux5ea6}{%
\subsection{Section 6: Education 教育, Educational diffusion rates
教育普及度:}\label{section-6-education-ux6559ux80b2-educational-diffusion-rates-ux6559ux80b2ux666eux53caux5ea6}}

The variables covered in this section are:

1.1 Educational diffusion rate (kindergartens)\\
1.2 Educational diffusion rate (day nurseries)

2.1 Working rate of day nurseries against the capacity

3.1 Ratio of long-term absentees from elementary school (30 days and
more for a school year) (per 1,000 pupils)\\
3.2 Ratio of long-term absentees from junior high school (30 days and
more for a school year) (per 1,000 students)\\
3.3 Ratio of long-term absentees from elementary school due to diseases
(30 days and more for a school year) (per 1,000 pupils)\\
3.4 Ratio of long-term absentees from junior high school due to diseases
(30 days and more for a school year) (per 1,000 students)\\
3.5 Ratio of long-term absentees from elementary school pupils due to
refusal to attend school (30 days and more for a school year)(per 1,000
pupils)\\
3.6 Ratio of long-term absentees from junior high school students due to
refusal to attend school (30 days and more for a school year)(per 1,000
students)

4.1 Ratio of junior high graduates going to further education\\
4.2 Ratio of senior high graduates going to further education

5.1 Ratio of high school graduates entering colleges and universities in
the same prefecture

6.1 Ratio of people having completed up to elementary or junior high
school only\\
6.2 Ratio of people having completed up to senior high school only\\
6.3 Ratio of people having completed up to junior colleges or
equivalent\\
6.4 Ratio of people having completed up to colleges and universities

\hypertarget{subsections-1.1---1.2-3}{%
\subsubsection{Subsections 1.1 - 1.2:}\label{subsections-1.1---1.2-3}}

They contain information on the Educational diffusion rate (\%)
教育普及度 for kindergartens 幼稚園, and day nurseries 保育所.

\begin{itemize}
\item
  The years covered are 2005, 2010-2018, 2020
\item
  The calculations are made by employing the formula: (Number of
  graduates) / (Number of first-grade elementary school students) * 100
\item
  For each prefecture and year, they would have: counted the number of
  kindergarten graduates and day nursery graduates, counted the number
  of first-grade elementary school students, divided the number of
  graduates from each type of institution by the number of first-grade
  students, then multiplied the result by 100 to get the percentage.
\item
  Both measures use the number of first-grade elementary school students
  as the denominator, which represents the total cohort of children
  entering formal education. These rates help to understand: the
  prevalence of early childhood education in each prefecture, the
  balance between kindergartens and day nurseries in providing
  pre-elementary education, and trends in early childhood education
  participation over time (as data is provided for 2005, 2010, and
  2011). The comparison between these two rates can also give insights
  into the preferences or availability of different types of early
  childhood education in various regions of Japan.
\end{itemize}

\hypertarget{subsection-2.1}{%
\subsubsection{Subsection 2.1:}\label{subsection-2.1}}

It contains information on the Working rate of day nurseries against the
capacity 保育所利用率 (\%).

\begin{itemize}
\item
  The years covered are 2005, 2010-2018, 2020
\item
  The calculations are made by employing the formula: (Number of
  children in day nurseries) / (Capacity of day nurseries) * 100.
\item
  For each prefecture and year, they would have: counted the total
  number of children currently enrolled in day nurseries, determined the
  total capacity of all day nurseries in the prefecture, divided the
  number of enrolled children by the total capacity, then multiplied the
  result by 100 to get the percentage.
\item
  This rate provides insights into the demand for day nursery services
  in each prefecture, how efficiently the existing day nursery
  infrastructure is being used, and potential shortages or surpluses in
  day nursery capacity. A rate close to or exceeding 100would indicate
  high demand and potentially a need for more day nursery facilities,
  while a lower rate might suggest underutilization of existing
  facilities.
\end{itemize}

\hypertarget{subsections-3.1---3.6}{%
\subsubsection{Subsections 3.1 - 3.6:}\label{subsections-3.1---3.6}}

They contain information on the Ratio of long-term absentees from
elementary school 小学校長期欠席児童比率 (per 1,000 pupils), Ratio of
long-term absentees from junior high school 中学校長期欠席生徒比率 (per
1,000 students), Ratio of long-term absentees from elementary school due
to diseases 病気による小学校長期欠席児童比率 (per 1,000 pupils), Ratio
of long-term absentees from junior high school due to diseases
病気による中学校長期欠席生徒比率 (per 1,000 students), Ratio of
long-term absentees from elementary school due to school refusal
不登校による小学校長期欠席児童比率 (per 1,000 pupils), Ratio of
long-term absentees from junior high school due to school refusal
不登校による中学校長期欠席生徒比率 (per 1,000 students).

\begin{itemize}
\item
  The years covered are 2005, 2010-2018, 2020-2021
\item
  The calculations are made by employing the formula: (Number of
  long-term absentees) / (Total number of students) * 1,000
\item
  For each prefecture and year, they would have: counted the number of
  long-term absentees for each category (total, due to illness, due to
  school refusal) for both elementary and junior high schools, divided
  these numbers by the total number of students in elementary or junior
  high schools respectively, then multiplied the result by 1,000 to get
  the rate per 1,000 students. These rates allow for comparison of
  absenteeism across prefectures and over time, providing insights into
  overall attendance issues, health-related absences, and school refusal
  problems in different regions of Japan.
\end{itemize}

\hypertarget{subsections-4.1---4.2}{%
\subsubsection{Subsections 4.1 - 4.2:}\label{subsections-4.1---4.2}}

They contain information on the Ratio (\%) of junior high graduates
going to further education 中学校卒業者の進学率, and Ratio of senior
high graduates going to further education 高等学校卒業者の進学率.

\begin{itemize}
\item
  The years covered are 2005, 2010-2018, 2020-2021
\item
  The calculations are made by employing the formula: (Number of
  graduates going to further education) / (Total number of graduates) *
  100
\item
  For each prefecture and year, they would have: counted the number of
  junior high or senior high graduates going to further education,
  counted the total number of graduates from each level, divided the
  number going to further education by the total number of graduates,
  then multiplied the result by 100 to get the percentage.
\item
  These rates provide insights into the progression of students through
  the education system, indicating the proportion of students who
  continue their education after completing junior high and senior high
  school. They can reflect educational aspirations, access to higher
  education, and potentially economic factors influencing educational
  decisions in different regions of Japan.
\end{itemize}

\hypertarget{subsection-5.1}{%
\subsubsection{Subsection 5.1:}\label{subsection-5.1}}

It contain information on the Ratio (\%) of high school graduates
entering colleges and universities in the same prefecture
出身高校所在地県の大学への入学者割合.

\begin{itemize}
\item
  The years covered are 2005, 2010, 2012-2022
\item
  The calculations are made by employing the formula: (Number of high
  school graduates entering colleges/universities in the same
  prefecture) / (Total number of high school graduates entering
  colleges/universities) * 100
\item
  For each prefecture and year, they would have: counted the number of
  high school graduates entering colleges/universities in the same
  prefecture, counted the total number of high school graduates entering
  colleges/universities anywhere, divided the number entering
  in-prefecture institutions by the total number entering higher
  education, then multiplied the result by 100 to get the percentage.
\item
  This rate provides insights into the retention of students within
  their home prefectures for higher education. It can reflect factors
  such as the quality and variety of higher education options within
  each prefecture, student preferences for staying close to home or
  moving away for college, and potentially economic factors influencing
  these decisions.
\end{itemize}

\hypertarget{subsections-6.1---6.4}{%
\subsubsection{Subsections 6.1 - 6.4:}\label{subsections-6.1---6.4}}

They contain information on the Ratio of people having completed up to
elementary or junior high school 最終学歴が小学・中学卒の者の割合, Ratio
of people having completed up to senior high school
最終学歴が高校・旧中卒の者の割合, Ratio of people having completed up to
junior colleges or equivalent 最終学歴が短大・高専卒の者の割合, and
Ratio of people having completed up to colleges and universities
最終学歴が大学・大学院卒の者の割合.

\begin{itemize}
\item
  The years covered are 1990, 2000, 2010, 2020
\item
  The calculations are made by employing the formula: (Number of people
  with specific education level) / (Total number of graduates) * 100
\item
  For each prefecture and year, they would have: counted the number of
  people whose highest level of education completed was
  elementary/junior high school, senior high school, junior
  college/technical college, or university/graduate school; counted the
  total number of graduates; divided the number for each education level
  by the total number of graduates; then multiplied the result by 100 to
  get the percentage.
\item
  These rates provide a comprehensive picture of the educational
  attainment levels across different regions of Japan over time. They
  can reflect changes in access to education, societal expectations for
  education, economic factors influencing educational decisions, and
  potentially the changing demands of the job market in different areas.
  The comparison across years (1990, 2000, 2010) allows for analysis of
  trends in educational attainment over two decades.
\end{itemize}

\hypertarget{section-7-education-ux6559ux80b2-education-expenditure-per-pupil-1ux4ebaux5f53ux305fux308aux306eux5b66ux6821ux7b49ux6559ux80b2ux8cbb}{%
\subsection{Section 7: Education 教育, Education expenditure per pupil
1人当たりの学校等教育費:}\label{section-7-education-ux6559ux80b2-education-expenditure-per-pupil-1ux4ebaux5f53ux305fux308aux306eux5b66ux6821ux7b49ux6559ux80b2ux8cbb}}

The variables covered in this subsection are:

1.1 Elementary school educational expenses (per pupils)\\
1.2 Junior high school educational expenses (per students)\\
1.3 High school educational expenses (Full-time senior) (per students)\\
1.4 Kindergarten educational expenses (per Kindergarten pupils)\\
1.5 Authorized child centers educational expenses
(Kindergarten-and-day-care center type) (per pupils)

\hypertarget{subsections-1.1---1.5-1}{%
\subsubsection{Subsections 1.1 - 1.5:}\label{subsections-1.1---1.5-1}}

They contain information on Elementary school educational expenses
小学校教育費 (per pupils, in yen) (児童1人当たり, 円), Junior high
school educational expenses 中学校教育費 (per students, in yen)
(生徒1人当たり, 円), High school educational expenses {[}Full-time
senior{]} 高等学校教育費[全日制](per students, in yen)
(生徒1人当たり,円), Kindergarten educational expenses 幼稚園教育費 (per
Kindergarten pupils, yen)(在園者1人当たり, 円), Authorized child
centers educational expenses {[}Kindergarten-and-day-care center type{]}
幼保連携型認定こども園教育費 (per pupils, yen)(在園者1人当たり, 円).

\begin{itemize}
\item
  The years covered are 2000, 2005, 2010-2017, 2019 for Elementary
  school educational expenses
\item
  The years covered are 2000, 2005, 2010-2017, 2019-2020 for Junior high
  school educational expenses and High school educational expenses
\item
  The years covered are 2000, 2005, 2010-2015, 2020 for Kindergarten
  educational expenses
\item
  The years covered are 2015-2017, 2019-2020 for Authorized child
  centers educational expenses
\item
  The calculations are made by employing the formula: (Total educational
  expenses for the specific school type) / (Total number of students in
  that school type)
\item
  For each prefecture and year, they would have: calculated the total
  educational expenses for each school type (elementary, junior high,
  high school, kindergarten, authorized child centers), counted the
  total number of students in each school type, then divided the total
  expenses by the number of students to get the per-pupil expenditure.
\end{itemize}

According to the data dictionary: - For elementary schools (E8102),
junior high schools (E8103), and high schools (E8104), the denominator
is listed as ``School education expenditure per student'' for each
respective school type. - For kindergartens (E8101) and authorized child
centers (E810101), the denominator is also listed as ``School education
expenditure per student'' for each respective type.

\begin{itemize}
\item
  These metrics provide insights into the financial investment in
  education across different levels of schooling and across different
  prefectures in Japan. They allow for comparisons of educational
  spending between different types of schools and between different
  regions. The data over time enables analysis of trends in educational
  funding, which could reflect changes in educational policies, economic
  conditions, or demographic shifts. Higher per-pupil expenditures might
  indicate greater resources available for education, potentially
  leading to better facilities, smaller class sizes, or more
  comprehensive educational programs. However, it's important to
  consider these figures alongside other metrics, such as educational
  outcomes and regional economic factors, for a more complete
  understanding of the education system's efficiency and effectiveness.
\item
  The inclusion of authorized child centers (Kindergarten-and-day-care
  center type) in this category reflects Japan's evolving early
  childhood education and care system, integrating traditional
  kindergartens with day-care services. This allows for analysis of
  resource allocation in these integrated facilities compared to
  traditional kindergartens and other school types.
\item
  When interpreting these statistics, researchers and policymakers
  should be aware of potential differences in cost structures between
  urban and rural areas, variations in local funding models, and any
  changes in the definition or scope of educational expenses over time.
  Cross-referencing these expenditure figures with other educational and
  demographic data can provide a more comprehensive picture of the
  educational landscape across Japan.
\end{itemize}



\end{document}
